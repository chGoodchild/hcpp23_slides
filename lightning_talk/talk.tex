\documentclass{beamer}
\usetheme{DarkConsole}
\usepackage{circuitikz}
\usetikzlibrary{arrows}

% Define a subcircuit 'divider' with input and output
\ctikzsubcircuitdef{divider}{in, out}{%
  coordinate (#1-in) to[R, l=$Z_s$, name=#1-rh, -*] ++(2,0)
  coordinate(#1-tmp) to[R, l=$Z_L$, name=#1-rl] ++(0,-2)
  node[tlground]{} (#1-tmp) --++(0.5,0) coordinate(#1-out)
}

% Define a custom command to use the 'divider' subcircuit
\newcommand{\mydiv}[4]{
  \divider{#1}{#2} (#1-rh.n) node[above] {$#3$}
  (#1-rl.n) node[right] {$#4$} (#1-out)
}


% Title page information
\title{Your Presentation Title}
\author{Your Name}
\date{\today}

\begin{document}

\begin{frame}
  \titlepage
\end{frame}

\begin{frame}{Outline}
  \tableofcontents
\end{frame}

\section{Introduction}

\begin{frame}{Introduction}
  This is the introduction slide.
\begin{tikzpicture}
  % Draw the subcircuit 'divider' twice with different labels
  \draw (0,0) \mydiv{a}{in}{}{};
  \draw (a-out) -- \mydiv{b}{in}{}{};
\end{tikzpicture}


\end{frame}

\section{Main Content}

\begin{frame}{Main Content}
  \begin{itemize}
    \item Bullet point 1
    \item Bullet point 2
    \item Bullet point 3
  \end{itemize}
\end{frame}

\begin{frame}{Main Content}
  \begin{enumerate}
    \item Item 1
    \item Item 2
    \item Item 3
  \end{enumerate}
\end{frame}

\section{Conclusion}

\begin{frame}{Conclusion}
  This is the conclusion slide.
\end{frame}

\begin{frame}{Thank You}
  \centering
  \Huge Thank you for your attention!
\end{frame}

\end{document}
