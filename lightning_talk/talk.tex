\documentclass{beamer}
\usetheme{DarkConsole}
\usepackage{circuitikz}
\usetikzlibrary{arrows}

% Title page information
\title{Your Presentation Title}
\author{Your Name}
\date{\today}

\begin{document}

\begin{frame}
  \titlepage
\end{frame}

\begin{frame}{Outline}
  \tableofcontents
\end{frame}

\section{Introduction}

\begin{frame}{Introduction}
  This is the introduction slide.

  \begin{center}

\begin{circuitikz}[american, scale=0.7, transform shape]
  \draw
  (0,0) to[R, l=$Z_L$] (0,3)
  to[short] (3,3)
  to[R, l=$Z_s$] (3,0)
  to[short] (0,0)

  (3,3) to[short, -o] (5,3) node[above] (a) {a}
  (3,0) to[short, -o] (5,0) node[below] (b) {b}
  ;

  \draw[>=angle 90,<->,red,
        shorten <=1mm, shorten >=1mm] (a) to node [fill=white] {$V_L$} (b);

  \draw[->, red] (0,3) -- (0,0) node[midway, left] {$V_s$};
\end{circuitikz}

	\begin{circuitikz}
		% Coordinates
		\coordinate (A) at (0,0);
		\coordinate (B) at (0,3);
		\coordinate (C) at (6,3);
		\coordinate (D) at (6,0);

		% Circuit
		\draw (A) to[R, l_=$Z_s$, v^<, name=z_s] (B)
		      to[R, l_=$Z_L$, v^<, name=z_l, *-*] (C)
		      to[short, -o] (D);

		% Voltages
		\fixedvlen[0.4cm]{z_s}{$V_s$}
		\fixedvlen[0.6cm]{z_l}{$V_L$}
	\end{circuitikz}

\begin{circuitikz}[american, scale=0.7, transform shape]
  \draw
  (0,0) to[R, l=$Z_L$] (0,3)
  to[short] (3,3)
  to[R, l=$Z_s$] (3,0)
  to[short] (0,0)

  (3,3) to[short, -o] (5,3) node[above] (a) {a}
  (3,0) to[short, -o] (5,0) node[below] (b) {b}
  ;

  \draw[>=angle 90,<->,red,
        shorten <=1mm, shorten >=1mm] (a) to node [fill=white] {$V_L$} (b);

  \draw[->, red] (0,3) -- (0,0) node[midway, left] (a1) {$V_s$};
  \node[above=0.3cm of a1] (a2) {a};
  \node[below=0.3cm of a1] (b1) {b};
\end{circuitikz}

\begin{circuitikz}[american, scale=0.7, transform shape]
  \draw
  (0,0) to[R, l=$Z_L$] (0,3)
  to[short] (3,3)
  to[R, l=$Z_s$] (3,0)
  to[short] (0,0)

  (3,3) to[short, -o] (5,3) node[above] (a) {a}
  (3,0) to[short, -o] (5,0) node[below] (b) {b}
  ;

  \draw[>=angle 90,<->,red,
        shorten <=1mm, shorten >=1mm] (a) to node [fill=white] {$V_L$} (b);

  \draw[->, red] (0,3) -- (0,0) node[midway, left] (a2) {$V_s$};

  % Adding points 'a' and 'b' on both sides of the 'Vs' arrow
  \node at (-1,1.5) (a1) {a};
  \node at (6,1.5) (b1) {b};
\end{circuitikz}
  \end{center}

\end{frame}

\section{Main Content}

\begin{frame}{Main Content}
  \begin{itemize}
    \item Bullet point 1
    \item Bullet point 2
    \item Bullet point 3
  \end{itemize}
\end{frame}

\begin{frame}{Main Content}
  \begin{enumerate}
    \item Item 1
    \item Item 2
    \item Item 3
  \end{enumerate}
\end{frame}

\section{Conclusion}

\begin{frame}{Conclusion}
  This is the conclusion slide.
\end{frame}

\begin{frame}{Thank You}
  \centering
  \Huge Thank you for your attention!
\end{frame}

\end{document}
